\chapter{Einleitung}

Im folgenden werden die Ergebnisse des Compilerbau Projektes
der Vorlesung \enquote{Spezielle Kapitel der Praktischen Informatik: Compilerbau, INF3199A}
dokumentiert.
Ziel war es einen Java Compiler zu entwickeln der eine Teilmenge der Programmiersprache
Java \cite{aufgabenstellung} zu Java Bytecode\cite{bytecode} compiliert.
Der Bytecode kann anschließend mit einer \ac{JVM}\cite{jvm} ausgeführt werden.
Des weiteren wurde ein Testframework entwickelt um die Teilfunktionen
des Compilers zu validieren.

Dieses Dokument dient als Dokumentation des Compilers, um einen groben
Überblick der Struktur des Projektes zu bekommen, den Compiler zu bauen,
das Testframework auszuführen und zur Dokumentation der Aufgabenverteilung.

Das Projekt wurde innerhalb des Wintersemesters 17/18 an der Universität
Tübingen in einer Gruppe von 5 Leuten implementiert.
Hierfür wurde das Projekt in Parser/Lexer, Typchecker, Codegenerierer und Tester unterteilt.
