\chapter{Architektur}
Das folgende Kapitel ist eine Übersicht der Architektur des Projektes.
Es hat \textbf{nicht} den Anspruch eine Technische Dokumentation der Architektur zu sein,
sondern einen Überblick der Komponenten des Projektes zu verschaffen.

\textit{Bemerkung: } Im folgenden werden alle Pfade relativ zu \verb+compilerbau-WS17-18/project+
angegeben.

Insgesamt besteht das Projekt aus drei Unterprogrammen.
Um maximale Widerverwendung zu gewährleisten wurden die Kernfunktionen
in einer Bibliothek (Library) zusammengefasst.
Diese Bibliothek ist unter \verb+src+ zu finden.
Sie besteht wiederum aus folgenden Komponenten:

\begin{enumerate}
	\item Lexer
	\item Parser
	\item Typchecker
	\item Codegenerierer
\end{enumerate}

Des weiteren wurde ein \ac{TUI} entwickelt, womit die Funktionen
des Compilers ausgeführt werden können.
Da die Funktionen in der Bibliothek implementiert sind,
linked das \ac{TUI} gegen die Bibliothek.
Es basiert auf der argparser Bibliothek \footnote{\url{https://hackage.haskell.org/package/argparser-0.3.4/docs/System-Console-ArgParser.html}} und ist unter \verb+cli/main.hs+ zu finden.

Das Letzte Unterprogramm ist das Testframework, welches
unter \verb+test+ zu finden ist und gleichermaßen wie das \ac{TUI} gegen
die Bibliothek linked.

\section{Implementierung der Hauptbibliothek}

Ziel der Entwicklung der Hauptbibliothek war es
einzelne Funktionen für das Lexen, Parsen, etc. zur Verfügung zu stellen.
Als gemeinsame Datenstruktur wurde eine abstrakte Syntax verwendet.
Die abstrakte Syntax ist unter \verb+src/ABSTree.hs+ zu finden.
Des weiteren gibt es eine \lstinline{Lexer.lex} methode, welche eine Liste an Tokens erzeugt.
Anschließend erstellt der Parser mit \lstinline{Parser.parse} aus dieser
Liste eine abstrakte Syntax.
Allgemein ist eine abstrakte Syntax eine Liste von Klassen.
Da Java eine typisierte Sprache ist transformiert der Typchecker die
abstrakte Syntax zu einer getypten abstrakten Syntax mittels \lstinline{Typechecker.checkTypes}.
Der Codegenerator erstellt dann mit \lstinline{Codegen.GenerateClassFile.genClass} anhand der getypten abstrakten Syntax den Bytecode.

Die Implementierung des jeweiligen Schrittes ist unter den Ordnern aus \autoref{paths} zu finden.

\begin{table}[H]
	\centering
	\begin{tabular}{l|l}
		\textbf{Komponente} & \textbf{Pfad}\\
		\hline
		Lexer & \verb+src/Lexer/Lexer.x+\\
		Parser & \verb+src/Parser/Parser.y+\\
		Typchecker & \verb+src/TypeChecker.hs+\\
		Codegenerator & \verb+src/Codegen/*+\\
	\end{tabular}
	\caption{Quellcode pro Komponente}\label{paths}
\end{table}

\section{Implementierung des Testframework}

\todo{Beschreibe einen Testaufbau}
\todo{Wo finde ich den Quellcode}
\todo{Beispiel zum kompilieren der Primzahlen}
