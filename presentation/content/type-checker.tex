\section{Typechecker}
\begin{frame}
    \frametitle{Typechecker - Prinzip}

Program liegt als Syntaxbaum (Klassenliste) vor

Der Typechecker:
\begin{itemize}
	\item arbeitetet diesen Baum rekursiv ab (bottom-up)
	\pause
	\item umschließt Expressions/Statements mit Typlabel (string)
	\pause
	\item nutz lookup-Tabellen zur Prüfung der Sichtbarkeit
	\pause
	\item prüft, ob Typen an entsprecheden Stellen zusammenpassen
\end{itemize}
\pause
Typen: int, boolean, char, class-types
\end{frame}

\begin{frame}
    \frametitle{Typechecker - Klassenliste}
${\bf C} = [c_1, \cdots , c_n]$ \\
Iteriere über $C$, für jedes $c \in C$:
\pause
\begin{itemize}
    \item Iteriere über Variablendeklarationen
    \begin{itemize}
        \item Überprüfe Initialdeklarationen der Variablen 
        \item konstruiere (Name:Typ)-Tabelle ${\bf L}$ der Feldvariablen
        \item Klassentyp $"$this$"$
    \end{itemize}
    \pause
    \item Iteriere über Methodendeklarationen
    \begin{itemize}
        \item Prüfe ob Bodytyp (Syntaxbaum) und ReturnTyp zusammenpassen
    \end{itemize}
\end{itemize}
\pause
$\Rightarrow$ Syntaxbaum wird rekursiv abgearbeitet mit ${\bf C}$, ${\bf L}$
\end{frame}

\begin{frame}[fragile]
    \frametitle{Typechecker - Blocks}
Funktionsbody liegt als Block statement (Statemensequenz) vor \\
Returntyp $\rightarrow$ Bodytyp \\
\pause
Handling: fold-Operation wird auf Block ausgeführt
\begin{lstlisting}[language=Haskell]
{ int x = 2; int y = 3; int res = x + y; return res; }	
\end{lstlisting}
\pause
($\{ \}$, void, $L$) \\
\pause
$\Rightarrow$ ($\{"x=2"\}$, void, $L \leftarrow (x,int)$) \\
\pause
$\Rightarrow$ ($\{"x=2","y=3"\}$, void, $L \leftarrow (x,int),(y,int)$) \\
\pause
$\Rightarrow$ ($\{"x=2","y=3","res=x+y"\}$, void, $L \leftarrow (x,int),(y,int),(res,int)$) \\
\pause
$\Rightarrow$ ($\{"x=2","y=3","res=x+y","return"\}$, \bf{int}, $L \leftarrow (x,int),(y,int),(res,int)$) \\
\end{frame}

\begin{frame}
    \frametitle{Typchecker - Expressions (Auszug)}
Klassenliste $C$, Variablenliste $L$ \\
- Literale: True\\
(Booleanliteral True) $\Rightarrow$ ({\bf TypedExpr} (Booleanliteral True) $"$boolean$"$) \\
\pause
- Variablenaufruf: x \\
(LocalOrFieldVar x) \\
$\Rightarrow$ Lookup von x in ${\bf L}$ $\rightarrow$ char \\
$\Rightarrow$ ({\bf TypedExpr} (LocalOrFieldVar x) $"$char$"$) \\
\pause
- Instanzvariablen: x.y \\
(InstVar (x:A) y) \\
$\Rightarrow$ Lookup von A in {\bf C} \\ 
$\Rightarrow$ Lookup von y in A $\rightarrow$ int \\
$\Rightarrow$  ({\bf TypedExpr} (InstVar (x:A) y) $"$int$"$)
 
\end{frame}

\begin{frame}
    \frametitle{Typechecker - Expressionstatements (Auszug)}
Klassenliste $C$, Variablenliste $L$ \\
- Methodenaufruf: x.f(2, 3) \\
(MethodCall (x:A) f (2:int, 3:int)) \\
$\Rightarrow$ Lookup von A in ${\bf C}$ \\
$\Rightarrow$ von f in A $\rightarrow$ returnTyp boolean\\
$\Rightarrow$ Unifikation gegebener und verlangter Argumente \\
$\Rightarrow$ ({\bf TypedStmtExpr} (MethodCall (x:A) f (2:int, 3:int)) $"$boolean$"$) \\
\ \\
\pause
Noch nicht klar, ob als Statement (Typ void) oder als Expression verwendet
\end{frame}

\begin{frame}
    \frametitle{Typechecker - Statements (Auszug)}
Klassenliste $C$, Variablenliste $L$ \\
- Blocks $\{ \cdots \}$\\
Gleiches vorgehen wie bei Funktionsbodys \\
Veränderte Variablenliste $L$ wird verworfen (scope) \\
Bodytype wird auch Bodytype des umgebenden Blocks \\
(Block [$\cdots$]) $\Rightarrow$ ({\bf TypedStmt} (Block [$\cdots$]) blocktype) \\
\ \\
\pause
- Variablendeklarationen: int i = 5;\\
Wie in Felddefinitionen für Klassen \\
${\bf L}$ wird aktualisiert zurückgegeben (shadowing) \\
(LocalVarDecls [$\cdots$]) $\Rightarrow$ ({\bf TypedStmt} (LocalVarDecls [$\cdots$]) $"$void$"$)
\end{frame}