\section{Code Generator}

\begin{frame}
        \frametitle{Module}
        Die folgenden Module werden für die Erzeugung des Class files aus dem ABSTree benutzt
        \begin{itemize}
                \item genClassFile.hs
                \item genConstantPool.hs
                \item genFields.hs
                \item genMethods.hs
        \end{itemize}
\end{frame}
\begin{frame}
        \frametitle{Module}
        In den nachfolgenden Modulen sind die Datentypen enthalten die für den abstrakten Bytecode benutzt werden
        \begin{itemize}
                \item Data/Assembler.hs
                \item Data/ClassFile.hs
        \end{itemize}
        Aus dem abstrakten Bytecode wird im Module Module BinaryClass.hs der Bytecode erzeugt
\end{frame}
\begin{frame}
        \frametitle{Constanten Pool}
        Der Constantenpool ist in einer hashMap die ein Eintrag auf dessen Position abbildet.

        Im Module genConstantPool.hs sind Funktionen enthalten die ein Eintrag erzeugen und dessen Position zurückgeben bzw nur die Position zurückgeben.
\end{frame}
\begin{frame}[fragile]
        \frametitle{Beispiel genConstantPool}
        \begin{lstlisting}
genMethodRefSuper :: String
                  -> Type
                  -> State ClassFile IndexConstantPool
genMethodRefSuper name typ =
  do indexClassName <- view (super . indexSp) <$> get
     indexNameType <- genNameAndType name typ
     genInfo MethodRefInfo
               { _tagCp              = TagMethodRef
               , _indexNameCp        = indexClassName
               , _indexNameandtypeCp = indexNameType
               , _desc               = ""
               }
        \end{lstlisting}
\end{frame}
\begin{frame}[fragile]
        \frametitle{Generieren der Methoden}
        Bei der Generierung der Methoden werden auch gleichzeitig die Einträge im constanten pool erstellt.  Im State wird folgender Datentyp verwendet.
        \begin{lstlisting}
data Vars
  = Vars { _localVar :: [HM.HashMap LocVarName LocVarIndex]
         , _allLocalVar :: S.Set LocVarName
         , _classFile :: ClassFile
         , _curStack :: Int
         , _maxStack :: Int
         , _line :: LineNumber
         , _continueLine :: [LineNumber]
         }
makeLenses ''Vars
        \end{lstlisting}
\end{frame}

\begin{frame}
        Als Major Version wird 48 anstatt 53 verwendet da die StackMapTable nicht implemntiert wurde
\end{frame}
